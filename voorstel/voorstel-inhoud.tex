%---------- Inleiding ---------------------------------------------------------

% TODO: Is dit voorstel gebaseerd op een paper van Research Methods die je
% vorig jaar hebt ingediend? Heb je daarbij eventueel samengewerkt met een
% andere student?
% Zo ja, haal dan de tekst hieronder uit commentaar en pas aan.

%\paragraph{Opmerking}

% Dit voorstel is gebaseerd op het onderzoeksvoorstel dat werd geschreven in het
% kader van het vak Research Methods dat ik (vorig/dit) academiejaar heb
% uitgewerkt (met medesturent VOORNAAM NAAM als mede-auteur).
% 

\section{Inleiding}
\label{sec:inleiding}

\subsection{Probleemstelling}
\label{sec:Probleemstelling}

Softwareprojecten binnen bedrijven en organisaties maken gebruik van talrijke externe bibliotheken, frameworks en tools, de zogenaamde \emph{dependencies}. Deze dependencies worden voortdurend bijgewerkt om nieuwe functionaliteiten, verbeterde prestaties en vooral beveiligingspatches aan te bieden \parencite{Pashchenko2018}.

Het manueel opvolgen van deze updates vormt echter een groot probleem in softwareontwikkeling. Ontwikkelaars moeten regelmatig changelogs doorzoeken, compatibiliteit controleren en inschatten welke updates veilig kunnen worden doorgevoerd, wat in de praktijk veel tijd vraagt en vaak wordt uitgesteld \parencite{Pashchenko2018,Behutiye2024}.

Onderzoek toont aan dat veel softwareprojecten hun dependencies niet systematisch bijhouden, waardoor een aanzienlijk deel verouderd raakt. Pashchenko et al.\ \parencite{Pashchenko2018} stellen bijvoorbeeld dat een belangrijk percentage kwetsbare dependencies relatief eenvoudig opgelost kan worden door een update, maar dat dit in de praktijk niet gebeurt wegens tijdsdruk en gebrek aan overzicht.

Wanneer het updateproces niet systematisch gebeurt, stapelen onderhoudstaken zich op, waardoor het steeds moeilijker en tijdrovender wordt om de software up-to-date te houden. Dit fenomeen staat bekend als \emph{technische schuld} (technical debt): de achterstand die ontstaat doordat noodzakelijke verbeteringen of updates te lang worden uitgesteld, wat leidt tot hogere onderhoudskosten en complexiteit op lange termijn \parencite{Behutiye2024,Ruiz2024}. Empirisch onderzoek bevestigt dat technische schuld binnen bedrijven reële negatieve gevolgen heeft voor productiviteit en softwarekwaliteit \parencite{Chalmers2020}.

Veel organisaties gebruiken al geautomatiseerde dependencybots, zoals Dependabot\footnote{Dependabot is de ingebouwde GitHub-tool voor het automatisch detecteren van kwetsbare of verouderde dependencies en het aanmaken van update-pull-requests.\autocite{GitHubDependabotDocs}} of Renovate\footnote{Renovate is een open-source dependencybot die geautomatiseerde update-pull-requests en uitgebreide configuratiemogelijkheden biedt voor diverse platformen en package managers.\autocite{RenovateDocs}}, die pull requests aanmaken zodra er nieuwe versies beschikbaar zijn \parencite{Erlenhov2022}. Deze oplossingen helpen om updates te detecteren, maar geven doorgaans weinig context over de inhoud en impact van de wijziging. Ontwikkelaars verliezen daardoor nog steeds tijd aan het interpreteren van changelogs, het inschatten van risico's en het beslissen of een pull request gemerged kan worden \parencite{Erlenhov2022}.

Binnen het bedrijf waar deze bachelorproef wordt uitgevoerd, leiden deze manuele analyses tot extra werkdruk voor een specifiek ontwikkelteam, namelijk het kleine kernteam dat verantwoordelijk is voor platform- en dependencybeheer over meerdere projecten en repositories heen. Daardoor gaat er disproportioneel veel tijd naar het nakijken van dependency-updates, ten koste van andere ontwikkeltaken.

Deze bachelorproef richt zich daarom op het ontwikkelen van een AI-gedreven systeem dat bovenop bestaande tooling zoals Dependabot\footnote{Zie \autocite{GitHubDependabotDocs} voor een overzicht van de functionaliteit, configuratieopties en best practices van Dependabot.} draait. Het systeem leest automatisch de door Dependabot aangemaakte pull requests in, analyseert de bijhorende changelogs en release notes, vat de essentie in begrijpelijke taal samen en ondersteunt het nemen van een beslissing (bijvoorbeeld mergen, uitstellen of extra acties uitvoeren). De tool moet onder andere kunnen controleren welke dependency wordt bijgewerkt, welk type update het betreft (patch, minor, major), wat de belangrijkste wijzigingen zijn en moet, indien gewenst, automatisch een pull request goedkeuren of sluiten volgens vooraf gedefinieerde regels.

Op die manier kan het updateproces slimmer, sneller en inzichtelijker worden gemaakt voor een duidelijk afgebakende doelgroep: het kleine kernteam dat binnen het bedrijf verantwoordelijk is voor dependencybeheer en platformonderhoud, in plaats van alle ontwikkelteams in het algemeen.

\subsection{Onderzoeksvraag}
\label{sec:Onderzoeksvraag}

Om het probleem van handmatig dependencybeheer en beperkte context bij updates op te lossen, wordt de volgende hoofdonderzoeksvraag geformuleerd:

\begin{quote}
  Hoe kan een AI-gedreven agent worden ingezet om dependency-updates uit bestaande tools voor automatisch dependencybeheer te analyseren en te beheren, zodat het kernteam dat verantwoordelijk is voor platform- en dependencybeheer efficiënter en met minder handmatig werk dependency-updates kan verwerken?
\end{quote}

Hieruit vloeien de volgende deelvragen voort.

\subsubsection{Deelvragen voor het beter begrijpen van het probleem}
\label{sec:Deelvragen_voor_het_beter_begrijpen_van_het_probleem}

\begin{enumerate}
  \item Wat zijn de belangrijkste uitdagingen en risico's bij het huidige, overwegend manuele dependency management voor het kernteam dat verantwoordelijk is voor platform- en dependencybeheer?
  \item Welke bestaande tools en oplossingen voor automatisch dependencybeheer worden vandaag gebruikt, en welke sterke punten en beperkingen hebben deze oplossingen in de context van dit bedrijf?
  \item Hoe leidt ophopende technische schuld op het vlak van dependency-updates ertoe dat er steeds meer tijd en inspanning nodig zijn voor het controleren en bijwerken van dependencies binnen het bedrijf?
\end{enumerate}

\subsubsection{Deelvragen richting de oplossing}
\label{sec:Deelvragen_richting_de_oplossing}

\begin{enumerate}
  \setcounter{enumi}{3}
  \item Hoe kunnen AI-agents en workflow-automatiseringsplatformen worden ingezet om informatie uit changelogs, release notes en dependency-pull-requests automatisch te interpreteren en in begrijpelijke vorm te communiceren naar het verantwoordelijke kernteam?
  \item Welke functionele en niet-functionele vereisten moet een platform voor AI-gestuurd dependencybeheer vervullen, bijvoorbeeld op het vlak van integratie met versiebeheersystemen en registries, uitbreidbaarheid, beheer van regels en policies, auditlogging en performantie?
  \item In welke mate voldoen geselecteerde AI-agent- of workflowplatformen aan deze vereisten, en hoe goed kunnen zij geïntegreerd worden in een proof-of-concept voor dependency management in de context van het bedrijf?
\end{enumerate}

\subsection{Onderzoeksdoelstelling}
\label{sec:Onderzoeksdoelstelling}

Het doel van dit onderzoek is om een proof-of-concept te ontwikkelen van een webapplicatie die automatisch dependency-updates detecteert, analyseert en communiceert via een geïntegreerde AI-agent, en om na te gaan in welke mate geselecteerde AI-agent- of workflowplatformen voldoen aan de vereisten van het bedrijf voor geautomatiseerd dependencybeheer. Eerder onderzoek toont aan dat AI-technieken succesvol kunnen worden ingezet binnen onderhoudstaken zoals changelog-analyse en code review, wat aantoont dat deze toepassing haalbaar en relevant is \parencite{Behutiye2024,Joshi2025,Kim2024,Zhu2025}.

Concreet zal de oplossing:
\begin{itemize}
  \item per project de gebruikte dependencies bijhouden;
  \item automatisch nagaan of er nieuwe versies beschikbaar zijn via publieke registries (zoals npm\footnote{npm is het standaardpakketbeheerplatform voor het JavaScript-ecosysteem.\autocite{NpmHome}}, Maven Central\footnote{Maven Central is de centrale repository voor Java-artifacts in het Maven-ecosysteem.\autocite{MavenCentral}}, NuGet\footnote{NuGet is de officiële package manager en online galerij voor .NET-bibliotheken.\autocite{NuGetGallery}} en PyPI\footnote{PyPI is de officiële Python Package Index voor het publiceren en distribueren van Python-pakketten.\autocite{PyPI}}) of via bestaande tools zoals Dependabot\footnote{Zie opnieuw \autocite{GitHubDependabotDocs} voor de integratie van Dependabot met verschillende ecosystemen.};
  \item changelogs, release notes en dependency-pull-requests analyseren met behulp van een AI-agent;
  \item het verantwoordelijke kernteam informeren over de aard en impact van de updates, inclusief een korte, begrijpelijke samenvatting en een indicatie of de update veilig lijkt om door te voeren;
  \item regels implementeren om, waar mogelijk, automatisch een pull request goed te keuren of te sluiten, zodat het kernteam minder manuele beslissingen hoeft te nemen;
  \item geselecteerde platformen beoordelen op geschiktheid op basis van gebruiksgemak, integratiemogelijkheden, onderhoudbaarheid, performantie en mate waarin zij voldoen aan de gedefinieerde vereisten.
\end{itemize}

De tool zal opgebouwd worden met \texttt{Next.js}\footnote{Next.js is een React-gebaseerd webframework voor het bouwen van full-stack webapplicaties met server-side rendering en API-routes.\autocite{NextJS}} als frontend, een C\#\footnote{C\# is een moderne, objectgeori\"enteerde programmeertaal op het .NET-platform.\autocite{CSharpDotnet}} (.NET) of Java (Spring Boot\footnote{Spring Boot is een Java-framework dat het bouwen en configureren van productieklare microservices vereenvoudigt.\autocite{SpringBoot}}) backend, en een database (PostgreSQL\footnote{PostgreSQL is een open-source relationeel databasemanagementsysteem dat sterk inzet op betrouwbaarheid en uitbreidbaarheid.\autocite{PostgreSQL}}, MongoDB\footnote{MongoDB is een documentgeori\"enteerde NoSQL-database gericht op flexibiliteit en schaalbaarheid.\autocite{MongoDB}} of Supabase\footnote{Supabase is een ontwikkelplatform bovenop PostgreSQL dat out-of-the-box API's, authenticatie en opslag aanbiedt.\autocite{Supabase}}) voor gebruikers- en projectbeheer. Het onderzoek resulteert in:
\begin{itemize}
  \item een prototype (demo) dat de werking van het systeem aantoont binnen een of enkele projecten van het bedrijf;
  \item een evaluatie van de toegevoegde waarde van AI bij dependency management voor het specifieke kernteam;
  \item een aanbeveling welk type platform, en eventueel welke concrete technologie, het meest geschikt is voor verdere toepassing binnen het bedrijf.
\end{itemize}

\subsection{Opzet van deze bachelorproef}
\label{sec:Opzet_van_deze_bachelorproef}

De bachelorproef volgt een klassieke onderzoeksstructuur en is als volgt opgebouwd:

\begin{itemize}
  \item \textbf{Hoofdstuk 1 -- Inleiding en probleemstelling:}
        Dit hoofdstuk beschrijft de context, de probleemstelling, de onderzoeksvragen en de doelstellingen van de bachelorproef.

  \item \textbf{Hoofdstuk 2 -- Literatuurstudie:}
        In dit hoofdstuk wordt de relevante literatuur rond dependency management, bestaande tools en oplossingen, technische schuld en AI-agent-frameworks en workflow-automatisering geanalyseerd.

  \item \textbf{Hoofdstuk 3 -- Methodologie en implementatie:}
        Dit hoofdstuk bespreekt de onderzoeksaanpak, het ontwerp van de systeemarchitectuur, de selectie van platformen en de implementatie van de proof-of-concept.

  \item \textbf{Hoofdstuk 4 -- Verwachte resultaten:}
        In dit hoofdstuk worden de verwachte resultaten van de implementatie en experimenten gepresenteerd.

  \item \textbf{Hoofdstuk 5 -- Verwachte conclusie:}
        Dit hoofdstuk formuleert de verwachte belangrijkste bevindingen op basis van de onderzoeksvragen.
\end{itemize}


%---------- Stand van zaken ---------------------------------------------------

\section{Literatuurstudie}
\label{sec:literatuurstudie}

\subsection{Dependency management en kwetsbare dependencies}

Moderne softwareprojecten steunen sterk op open-source dependencies, waardoor kwetsbaarheden in externe bibliotheken rechtstreeks een impact kunnen hebben op de veiligheid en betrouwbaarheid van toepassingen.\autocite{Pashchenko2018} Pashchenko et al.\ tonen aan dat een groot deel van de aangetroffen kwetsbare dependencies in de praktijk relatief eenvoudig verholpen kan worden door te upgraden naar een veiligere versie, maar dat ontwikkelteams deze updates vaak niet consequent doorvoeren.\autocite{Pashchenko2018}

In hun proefondervindelijk onderzoek onderscheiden Pashchenko et al.\ tussen daadwerkelijk gedeployde kwetsbare dependencies en libraries die enkel tijdens ontwikkeling of testing gebruikt worden, en stellen zij vast dat het merendeel van de reëel kwetsbare dependencies door de eigen ontwikkelaars of directe onderhouders kan worden opgelost.\autocite{Pashchenko2018} Dit onderstreept het belang van systematisch dependencybeheer en duidelijke inzichten in welke dependencies effectief risico vormen in productie omgevingen.\autocite{Pashchenko2018}

\subsection{Technische schuld en onderhoudslast}

Het uitstellen van dependency-updates draagt bij aan de opbouw van technische schuld, waardoor onderhoud en toekomstige wijzigingen steeds meer inspanning vereisen.\autocite{Behutiye2024,Ruiz2024} Behutiye et al.\ analyseren de rol van technische schuld in agile omgevingen en beschrijven hoe keuzes die op korte termijn tijdswinst opleveren, op langere termijn leiden tot hogere kosten en complexere wijzigingen.\autocite{Behutiye2024}

Ruiz et al.\ bespreken in hun tertiaire studie dat technische-schuldmanagement de voorbije jaren een groeiend onderzoeksdomein is geworden, waarbij organisaties worstelen met het in kaart brengen, prioriteren en terugdringen van opgebouwde schuld.\autocite{Ruiz2024} Empirisch werk van Chalmers University of Technology bevestigt dat technische schuld niet louter een theoretisch concept is, maar in industriële context concreet voelbaar is in de vorm van verminderde productiviteit, hogere foutkans en vertragingen in softwarelevering.\autocite{Chalmers2020}

\subsection{Dependencybots en hun beperkingen}

Om de handmatige last rond dependencybeheer te verlagen, zetten veel projecten dependency management bots in, zoals Dependabot\footnote{Dependabot is de ingebouwde GitHub-tool voor het monitoren van kwetsbaarheden en het automatisch aanmaken van update-pull-requests voor ondersteunde ecosystemen.\autocite{GitHubDependabotDocs}} en Renovate\footnote{Renovate is een open-source dependencybot die automatisch pull requests genereert voor versie- en beveiligingsupdates, met uitgebreide configuratiemogelijkheden en ondersteuning voor meerdere platformen.\autocite{RenovateDocs}}.\autocite{Erlenhov2022} Erlenhov et al.\ voeren een kwantitatieve en kwalitatieve studie uit naar dependencybots in open-source systemen en stellen vast dat slechts een beperkt aantal van de geanalyseerde geautomatiseerde tools voldoet aan de criteria van volwaardige ``Devbots'', die autonoom bijdragen aan de codebasis.\autocite{Erlenhov2022}

Hun resultaten tonen dat projecten vaak experimenteren met meerdere bots en regelmatig wisselen, onder meer door problemen met ruis (te veel pull requests), beperkte configureerbaarheid en moeilijkheden om de impact van voorgestelde updates goed te begrijpen.\autocite{Erlenhov2022} Hoewel dependencybots het detecteren en aanmaken van update-pull-requests automatiseren, blijft de inhoudelijke beoordeling van changelogs, compatibiliteit en risico's grotendeels bij ontwikkelaars liggen, wat aansluit bij de probleemstelling van deze bachelorproef.\autocite{Erlenhov2022,Pashchenko2018}

\subsection{AI, agents en workflow-automatisering}

Recente literatuur verkent hoe AI-agents en multi-agentframeworks kunnen worden ingezet om complexe taken in softwareontwikkeling en data-intensieve workflows te coördineren.\autocite{Joshi2025,Kim2024,Zhu2025} Joshi biedt een overzicht van autonome systemen en collaboratieve AI-agentframeworks en benadrukt dat zulke agents vooral nuttig zijn in scenario's met veel herhalende beslissingen op basis van tekstuele en gestructureerde input, zoals logs, notificaties en rapporten.\autocite{Joshi2025}

Kim et al.\ introduceren met Bel Esprit een multi-agentframework voor het opbouwen van AI-model-pijplijnen, waarbij verschillende gespecialiseerde agents samenwerken om complexere taken modulair af te handelen.\autocite{Kim2024} Zhu et al.\ bestuderen in OAgents empirisch welke ontwerpkeuzes leiden tot effectieve agents, en bespreken onder meer het belang van duidelijke taakafbakening, robuuste toolintegratie en feedbackloops voor betrouwbare beslissingsondersteuning.\autocite{Zhu2025} Deze inzichten zijn relevant voor het ontwerp van een AI-agent die dependency-updates interpreteert, samenvat en beslissingsvoorstellen formuleert voor een onderhoudsteam.

Naast generieke agentframeworks beschrijft Wali et al.\ hoe workflow-automatisering met n8n\footnote{n8n is een open-source, low-code workflow-automatiseringstool waarmee integraties tussen API's, databronnen en notificatiesystemen visueel gemodelleerd en uitgevoerd kunnen worden.\autocite{Wali2025}} operationele efficiëntie kan verhogen door repetitieve, gegevensgedreven processen te coördineren over verschillende systemen heen.\autocite{Wali2025} Hoewel hun casus zich richt op een financiële context, illustreert de studie dat low-code workflowtools geschikt zijn om integraties met externe APIs, databronnen en notificatiekanalen te realiseren, wat ook essentieel is voor een geautomatiseerde dependency-updatepipeline.\autocite{Wali2025}

\subsection{Onderzoeksniche: AI-ondersteund dependencybeheer}

Samengenomen schetst de literatuur een duidelijk beeld: kwetsbare of verouderde dependencies vormen een reëel risico, maar kunnen vaak verholpen worden door updates die vandaag onvoldoende systematisch worden uitgevoerd.\autocite{Pashchenko2018,Behutiye2024} Tegelijk tonen studies naar technische schuld dat het structureel uitstellen van onderhoud, waaronder dependency updates, leidt tot een groeiende onderhoudslast en achterstand.\autocite{Behutiye2024,Ruiz2024,Chalmers2020}

Onderzoek naar dependencybots maakt duidelijk dat huidige oplossingen weliswaar updates detecteren en pull requests genereren, maar ontwikkelaars nog steeds belasten met het interpreteren van changelogs en het beoordelen van risico's en prioriteiten.\autocite{Erlenhov2022} De recente vooruitgang in AI-agents en workflow-automatisering suggereert dat een volgende stap mogelijk is: een systeem waarin een AI-agent niet alleen updates signaleert, maar ook de inhoud van release notes en changelogs analyseert, samenvat en contextafhankelijk advies geeft aan een klein kernteam dat verantwoordelijk is voor dependencybeheer.\autocite{Joshi2025,Kim2024,Zhu2025,Wali2025}

Deze bachelorproef positioneert zich precies in deze niche door een AI-gestuurde laag bovenop bestaande dependencybots te onderzoeken en te prototypen, met de expliciete focus op het verlagen van de manuele beoordelingslast en het beheersbaar houden van technische schuld rond dependencies in een bedrijfscontext.


%---------- Methodologie ------------------------------------------------------
\section{Methodologie}
\label{sec:methodologie}

Hier beschrijf je hoe je van plan bent het onderzoek te voeren. Welke onderzoekstechniek ga je toepassen om elk van je onderzoeksvragen te beantwoorden? Gebruik je hiervoor literatuurstudie, interviews met belanghebbenden (bv.~voor requirements-analyse), experimenten, simulaties, vergelijkende studie, risico-analyse, PoC, \ldots?

Valt je onderwerp onder één van de typische soorten bachelorproeven die besproken zijn in de lessen Research Methods (bv.\ vergelijkende studie of risico-analyse)? Zorg er dan ook voor dat we duidelijk de verschillende stappen terug vinden die we verwachten in dit soort onderzoek!

Vermijd onderzoekstechnieken die geen objectieve, meetbare resultaten kunnen opleveren. Enquêtes, bijvoorbeeld, zijn voor een bachelorproef informatica meestal \textbf{niet geschikt}. De antwoorden zijn eerder meningen dan feiten en in de praktijk blijkt het ook bijzonder moeilijk om voldoende respondenten te vinden. Studenten die een enquête willen voeren, hebben meestal ook geen goede definitie van de populatie, waardoor ook niet kan aangetoond worden dat eventuele resultaten representatief zijn.

Uit dit onderdeel moet duidelijk naar voor komen dat je bachelorproef ook technisch voldoen\-de diepgang zal bevatten. Het zou niet kloppen als een bachelorproef informatica ook door bv.\ een student marketing zou kunnen uitgevoerd worden.

Je beschrijft ook al welke tools (hardware, software, diensten, \ldots) je denkt hiervoor te gebruiken of te ontwikkelen.

Probeer ook een tijdschatting te maken. Hoe lang zal je met elke fase van je onderzoek bezig zijn en wat zijn de concrete \emph{deliverables} in elke fase?

%---------- Verwachte resultaten ----------------------------------------------
\section{Verwacht resultaat}
\label{sec:verwachte_resultaten}



%---------- Verwachte conclusies ----------------------------------------------
\section{Verwacht conclusie}
\label{sec:verwachte_conclusies}